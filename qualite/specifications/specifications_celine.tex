\documentclass[a4paper,12pt]{article}
\usepackage[francais]{babel}
\usepackage[utf8]{inputenc}
\usepackage[T1]{fontenc}
\usepackage[top=2.7cm, bottom=2.7cm, left=2.7cm, right=2.7cm]{geometry}
\usepackage{listings}
\usepackage{graphicx}
\usepackage{color}


%opening
\title{Projet d'Informatique Répartie - Tournois - Spécifications}
\author{Alexandre Brehmer, Christophe Cluizel, Anthony Courtin,\\Céline Leduc, Charlotte Touchard, Simon Wallon}


\begin{document}

\maketitle

\section{Principe général}
Il s'agit de développer un service de tournois de labyrinthes réparti. Les joueurs sont placés au centre d'un labyrinthe et doivent en sortir. La partie est terminée lorsque qu'un joueur y parvient. Il est alors déclaré vainqueur.

\section{Interface utilisateur}
Le joueur a deux possibilités :
\begin{itemize}
	\item il dispose d'une zone de texte afin de coder sa propre intelligence artificielle
	\item il joue manuellement avec les touches du clavier
\end{itemize}

\section{Gestion des parties}
L'utilisateur peut choisir de créer une partie. Dans ce cas il spécifie le nombre de joueurs maximum. Il peut également choisir de rejoindre une partie en attente.

Une partie en attente est une partie créée qui n'a pas été lancée par son créateur.

Un fois la partie lancée, elle n'est plus dans la liste des parties en attente et ne peut plus être rejointe.

Rappel : dès qu'un joueur atteint la sortie du labyrinthe il est déclaré vainqueur. La partie est alors terminée.

\section{Déplacements}
L'intelligence artificielle ou le joueur doivent soumettre au programme une des quatre directions suivantes : HAUT, BAS, GAUCHE, DROITE.

Si le temps de réponse du joueur dépasse 1 seconde alors le déplacement n'est pas pris en compte et le joueur reste immobile.

Idem si le déplacement mène le joueur dans un mur.

\section{Serveur}
Plusieurs parties se déroulent en simultané.

Lorsqu'un joueur rejoint une partie en attente, celle-ci est choisie aléatoirement parmi celles ayant des places disponibles.


\end{document}
