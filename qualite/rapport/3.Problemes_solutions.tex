% =========== Logique métier ===========
\section{Logique métier}

    % -------- RMI -----------
    \subsection{RMI}
        Un problème que nous avons eu au départ était la gestion des client-serveurs. En effet, le client joue également le rôle de serveur et le serveur également le rôle de client. Pour répondre à ce problème, nous avons mis en place une double gestion de stub, une côté client (JoueurInterface) et une côté serveur (PartieManager).

    % -------- Compilation et exécution depuis le code source -------
    \subsection{Compilation et exécution depuis le code source}
	\label{problemes_IA}

        Si un joueur décide d'implémenter une IA pour jouer au jeu Javabyrinthe, il faut que celle-ci soit compilée et exécutée à partir de notre code java. Nous voulions initialement compiler et exécuter le code de l'IA sur le serveur afin d'autoriser au joueur plus de liberté. En effet, le joueur pourrait ainsi choisir son langage de programmation, sans nécessairement posséder les outils de compilation et d'exécution associés. \\

		Un tel fonctionnement nécessite le lancement d'un processus (contenant en l'occurrence la commande \texttt{javac} et l'exécution de l'intelligence artificielle) depuis notre code principal java du serveur ainsi qu'un dialogue entre ces deux éléments. Au vu de la complexité et l'éloignement du problème par rapport à la thématique de l'informatique répartie, nous avons décidé de compiler et d'exécuter ce code sur la machine cliente. \\

		Le plus simple aurait été de demander à l'utilisateur du programme de coder entièrement un fichier JoueurIA.java contenant les méthodes adéquates et de le compiler à la main. Notre programme n'aurait plus qu'à instancier cet objet. Afin de ne pas perdre le travail déjà effectué précédemment sur les processus, nous sommes parvenus à un état intermédiaire où notre programme compile et exécute automatiquement sur le client le code de l'IA récupéré dans le fichier \texttt{MainJoueur.java}.


% =========== Interface ========
\section{Interface}
    \subsection{Récupération des informations de déplacement}
        Lors de l'appel à la fonction \verb|jouer| d'un joueur humain, l'utilisateur doit saisir une touche directionnelle afin d'indiquer son déplacement. Cette fonction est bloquante, c'est-à-dire que tant que l'utilisateur n'a rien saisi, la fonction ne retourne rien et continue de tourner, bloquant le reste du thread. Le problème récontré est que les informations de saisis de touches sont des évènements générés dans la fenêtre, et le joueur n'a pas connaissance de cette fenêtre (il ne possède pas d'instance de la classe).

        La solution apportée fut de créer un attribut statique qui contiendrait le déplacement du joueur, et de créer une fonction get statique dessus, permettant ainsi au joueur humain de récupérer l'information de saisie. La fonction statique est donc bloquante, initialise le déplacement comme une chaine vide, et attend que l'utilisateur saisisse une touche pour retourner le déplacement.

    \subsection{Affichage des informations sur les joueurs}
        Pour développer l'IHM, nous avions besoin d'afficher les informations sur les joueurs : leur pseudo et leur position sur la carte. Nous avons rencontré le problème suivant : lorsqu'un joueur rejoignait une partie, il possèdait toutes les informations relatives à cette partie ; toutefois, les autres joueurs ne recevait pas l'information qu'un nouveau joueur était connecté. Ainsi, le joueur qui créait la partie n'affichait que ses informations.

        Ce problème est lié au stockage de l'objet \verb|partie| dans l'objet \verb|joueur|. Le joueur possède en effet un attribut \verb|partie|, ce qui permet à l'IHM d'afficher toutes les informations d'une partie en ayant connaissance que du joueur. Lors de l'ajout d'une partie à un joueur, une \emph{copie} de la partie était faite dans le joueur, et non une référence. Ainsi lorsque la partie (celle du serveur) était modifié, i.e. quand un joueur était ajouté, les parties des joueurs n'étaient pas modifiées en conséquence, et seul le dernier joueur a rejoindre la partie possède toutes les informations.

        Pour résoudre ce problème, nous avons décidé de recopier la partie dans chaque joueur à chaque modification, permettant ainsi à chaque joueur de stocker une partie à jour.
