\documentclass{scrreprt}
\usepackage[utf8]{inputenc}
\usepackage[T1]{fontenc}
\usepackage{lmodern}
\usepackage[francais]{babel, varioref}
\usepackage{graphicx}
\usepackage{listings}
%\usepackage[pdfborder={0 0 0},
 %           pdfcreator={},
   %         pdfsubject={Projet Correcteur orthographique},
  %          pdftitle={Projet Correcteur orthographique}]{hyperref}
\usepackage{xspace}
\usepackage{amssymb}
\usepackage{calc}
\usepackage{listingsutf8}
\usepackage{color}
\usepackage{xcolor}

%%%%%%%%%%%%%%%%%%%%%
\usepackage{url}
\usepackage[top=2.1cm,bottom=2.2cm,left=2cm,right=2cm]{geometry}
\usepackage[final]{pdfpages}
%%%%%%%%%%%%%%%%%%%%%%%%%%%

%%%%%%%%%%% Pour sommaire cliquable %%%%%%%%%%
\usepackage{hyperref} % Créer des liens et des signets
\hypersetup{
colorlinks=true, %colorise les liens
breaklinks=true, %permet le retour à la ligne dans les liens trop longs
urlcolor= blue, %couleur des hyperliens
linkcolor= black, %couleur des liens internes
citecolor=black,  %couleur des références
}
%%%%%%%%%%%%%%%%%%%%%%%%%%%%%%%%%%%%%%%%%%%%


\newcommand\TODO[1]{\textcolor{red}{\textbf{#1}}}

%pour la coloration du code

\definecolor{colFond}{rgb}{0.8,0.9,0.9}
\definecolor{hellgelb}{rgb}{1,1,0.8}
\definecolor{colKeys}{rgb}{0,0,1}
\definecolor{colIdentifier}{rgb}{0,0,0}
\definecolor{colComments}{rgb}{0,0.5,0}
\definecolor{colString}{rgb}{0.62,0.12,0.94}

\lstset{
  language=Java,
  float=hbp,
  basicstyle=\ttfamily\small,
  identifierstyle=\color{colIdentifier},
  keywordstyle=\bf \color{colKeys},
  stringstyle=\color{colString},
  commentstyle=\color{colComments},
  columns=flexible,
  tabsize=3,
  frame=single,
  frame=shadowbox,
  rulesepcolor=\color[gray]{0.5},
  extendedchars=true,
  showspaces=false,
  showstringspaces=false,
  numbers=left,
  firstnumber=1,
  numberstyle=\tiny,
  breaklines=true,
  backgroundcolor=\color{hellgelb},
  captionpos=b,
}

\usepackage{templateINSA}
\initINSA

\title{Javabyrinthe}

\author{Alexandre \bsc{Brehmer}\\ Christophe \bsc{Cluizel} \\ Anthony \bsc{Courtin} \\ Céline \bsc{Leduc} \\ Charlotte \bsc{Touchard} \\ Simon \bsc{Wallon}}
\renewcommand\soustitre{Rapport}
\renewcommand\infoBig{Projet d'Informatique Répartie}
\renewcommand\infoSmall{}

\begin{document}
 \titleINSA{15}{images/page_couverture.png}{0}{0}{225}{\url{http://fvirtman.free.fr/progs/makelaby.png}{\textcolor{white}{makezine.com}}}

\tableofcontents

\chapter{Introduction}
	\section*{Contexte du projet}
    Dans le cadre de notre formation au sein du département Architecture des Systèmes d'Information, un cours d'Informatique Répartie est dispensé. La réalisation d'un projet par groupes de 5 ou 6, au cours du semestre, a pour objectif de concevoir et développer un système réparti permettant de mettre en pratique les connaissances acquises en cours et en TD Machine. Un certain nombre de séances de TD sont dédiées à la réalisation de ce projet mais celui-ci nécessite également une recherche de solutions non-abordées ainsi qu'un travail personnel supplémentaire.

    Une soutenance aura lieu, en présence des autres étudiants, composée d'une présentation et d'une démonstration (ou vidéo).

    Il est indispensable que la solution mise en place soit une application distribuée. Le présent document de spécifications a valeur contractuelle et doit être respecté pour la suite du projet.


% ----------- Principe général -----------
\section*{Principe général}
    L'objectif de notre projet est de développer un serveur de tournois permettant à des IA et à des joueurs humains de s'affronter dans une arène.

    Nous avons choisi de développer un système distribué permettant à des joueurs de s'affronter dans des tournois de labyrinthes. Les joueurs sont placés au centre d'un labyrinthe et doivent en sortir. La partie est terminée lorsque qu'un joueur y parvient. Il est alors déclaré vainqueur.


% ----------- Contenu du présent document -----------
\section*{Contenu du présent document}
	Ce document présente les choix de conception opérés pour réaliser ce projet. Afin de clarifier les spécifications développées précédemment, un diagramme de cas d'utilisation est présenté, ainsi qu'un diagramme d'interactions. Enfin, notre conception est matérialisée par un diagramme de classes.


\chapter{Document de spécifications mis à jour}
  Le document suivant reprend nos spécifications mises à jour suite à l'avancement du projet. Les modifications apportées seront signalées en vert et en rouge.
	\includepdf[pages=-]{specifications.pdf}
  \emph{Fin du document de spécifications.}

\chapter{Document de conception mis à jour}
  Le document suivant reprend notre conception mise à jour suite à l'avancement du projet. Les modifications apportées seront signalées en vert et en rouge.
	\includepdf[pages=-]{conception.pdf}
  \emph{Fin du document de conception.}

\chapter{Choix techniques}
	\section{Choix du langage de programmation}

	Nous avons développé l'application Javabyrinthe en langage \textbf{Java}. Depuis le début du projet, nous avions choisi ce langage car c'est celui que tous les membres de l'équipe du projet maîtrisent le mieux. Nous voulions éviter le plus possible les problèmes pouvant être générés par une trop faible connaissance du langage. Ainsi, la plupart des difficultés rencontrées ne sont donc pas dues au langage de programmation en lui-même, mais plutôt au domaine de l'informatique répartie. En effet, ce projet n'a pas pour but de nous entraîner à programmer mais plutôt de gérer les communications entre le serveur et ses clients.

	D'autre part, un langage orienté objet nous a paru adapté à notre projet. Nous avons pu utiliser différents objets, comme un objet Labyrinthe et un objet Joueur. Visualiser ces éléments en tant qu'objet est beaucoup plus simple pour comprendre le fonctionnement du projet et travailler en équipe.

	\paragraph*{}
	Un générateur automatique de labyrinthe a aussi été implémenté. Ce dernier était optionnel, mais obtenir à chaque partie un labyrinthe différent est attractif. Ce générateur a été programmé en \textbf{Scala}. Ce langage a été utilisé car la personne qui a réalisé le générateur avait une préférence pour celui-ci.


\section{Communication client-serveur}

	Dans notre document de spécifications, nous avions décidé que l'échange d'information entre le serveur et les clients se ferait à l'aide de \textbf{sockets}.

	Cependant, au fur et à mesure de l'avancement du projet, après la phase de conception, nous nous sommes rendus compte qu'utiliser \textbf{RMI} (Remote Method Invocation) serait probablement plus simple. Même si nous avions décidé de n'échanger que des chaînes de caractères, utiliser RMI permet aussi de distribuer un objet avec ses méthodes. Choisir d'utiliser RMI nous permettait donc de changer de stratégie et de travailler sur des objets distribués si nous n'arrivions pas à travailler en échangeant seulement des chaînes de caractères. Nous avons finalement choisi d'échanger des objets, et choisir RMI était donc une solution adaptée.

	L'autre raison qui nous a fait modifier ce point du document est qu'à l'époque de sa rédaction, presque aucun membre de l'équipe ne connaissait RMI. Après l'avoir découvert en cours et expérimenté à l'aide de travaux pratiques, nous avons décidé d'opter pour l'utilisation de RMI plutôt que de sockets, car son utilisation nous a paru plus simple, de plus haut niveau.

	Utiliser RMI implique de programmer en Java, mais cela n'a pas été une contrainte pour l'équipe puisque nous avions décidé d'utiliser ce langage.

	Pour finir, RMI permet de gérer les exceptions de manière transparente, ce qui est un de ses avantages par rapport aux sockets. Gérer les exceptions levées par la compilation du programme IA codé par le joueur par exemple est donc simplifié.


\section{Affichage}

	Dès le début du projet, nous avions tous en tête la forme que prendrait notre jeu: le labyrinthe serait vu du dessus et prendrait la forme d'une grille avec des cases pleines représentant les murs. A partir de cette idée, nous avons cherché comment faire apparaître cette interface.

	Il était bien entendu possible de dessiner le labyrinthe dans un terminal avec des pipes et des underscores mais nous souhaitions une interface utilisateur plus jolie et attrayante. Après quelques recherches et en visualisant des exemples d'utilisation de différentes bibliothèques, il s'est avéré que la bibliothèque Java \textbf{Slick2D} correspond le mieux à nos attentes.

  En effet, cette bibliothèque a été spécialement conçue pour construire l'interface de jeux vidéos. Elle permet bien de la construire comme nous l'avions imaginé dans le document de spécifications, et permet aussi de dessiner un labyrinthe facilement reconnaissable : des chemins blancs délimités par des traits noirs représentant les murs.

  De plus, la documentation très complète de cette bibliothèque ainsi que sa communauté assez active nous a permis d'utiliser rapidement cette bibliothèque afin de faire apparaître tous les éléments graphiques que nous souhaitions. Les tutoriels étaient nombreux et de notre niveau.


\chapter{Problèmes rencontrés \& solutions apportées}
  % =========== Logique métier ===========
\section{Logique métier}

    % -------- RMI -----------
    \subsection{RMI}
        Un problème que nous avons eu au départ était la gestion des client-serveurs. En effet, le client joue également le rôle de serveur et le serveur également le rôle de client. Pour répondre à ce problème, nous avons mis en place une double gestion de stub, une côté client (JoueurInterface) et une côté serveur (PartieManager).

    % -------- Compilation et exécution depuis le code source -------
    \subsection{Compilation et exécution depuis le code source}
	\label{problemes_IA}
	
        Si un joueur décide d'implémenter une IA pour jouer au jeu Javabyrinthe, il faut que celle-ci soit compilée et exécutée à partir de notre code java. Nous voulions initialement compiler et exécuter le code de l'IA sur le serveur afin d'autoriser au joueur plus de liberté. En effet, le joueur pourrait ainsi choisir son langage de programmation, sans nécessairement posséder les outils de compilation et d'exécution associés.\\
		
		Un tel fonctionnement nécessite le lancement d'un processus (contenant en l'occurrence la commande \texttt{javac} et l'exécution de l'intelligence artificielle) depuis notre code principal java du serveur ainsi qu'un dialogue entre ces deux éléments. Au vu de la complexité et l'éloignement du problème par rapport à la thématique de l'informatique répartie, nous avons décidé de compiler et d'exécuter ce code sur la machine cliente.\\
		
		Le plus simple aurait été de demander à l'utilisateur du programme de coder entièrement un fichier JoueurIA.java contenant les méthodes adéquates et de le compiler à la main. Notre programme n'aurait plus qu'à instancier cet objet. Afin de ne pas perdre le travail déjà effectué précédemment sur les processus, nous sommes parvenus à un état intermédiaire où notre programme compile et exécute automatiquement sur le serveur le code de l'IA récupéré dans le fichier \texttt{MainJoueur.java}.

\TODO{Christophe, Alexandre, Charlotte/Céline}

% =========== Interface ========
\section{Interface}
    \TODO{Simon et Anthony}


\chapter{Améliorations possibles}
	% =========== Logique métier ===========
\section{Logique métier}

	Comme décrit précédemment, dans la partie \ref{problemes_IA}, la compilation et l'exécution de l'intelligence artificielle fournie par le joueur est située dans un état intermédiaire. Une amélioration évidente consisterait à délocaliser l'ensemble des opérations concernant l'IA sur le serveur. Ceci impliquerait de transmettre le code sous la forme d'une chaîne de caractères. Le serveur serait ensuite en charge de créer un fichier JoueurIA.java contenant ce code et le compiler grâce à un processBuilder. L'instanciation de l'objet devra alors être opérée sur le serveur et l'objet ne serait plus réparti. \\

	Comme mentionné précédemment, notre application pourrait être étendue à différents langages de programmation puisque l'IA s'exécute dans un processus à part dialoguant avec le programme principal. Dans l'hypothèse ou cette exécution (et la compilation associée) s'effectue sur le serveur, l'utilisateur n'aurait même pas besoin de posséder les outils de compilation et d'exécution du langage qu'il aura choisi. Par extension, des IA codées dans différents langages pourraient alors s'affronter dans les labyrinthes.

% =========== Interface graphique ===========
\section{Interface graphique}

Dans notre idée de départ, afin de ressembler à CodinGame, nous voulions que l'utilisateur soit en mesure de taper le code de son IA directement dans l'interface graphique. L'ajout d'une telle fonctionnalité dans Slick2d ne s'avère pas approprié. En effet, ceci complexifie le développement de l'interface et la gestion d'événements et ne permet pas de conserver la coloration syntaxique propre au langage utilisé. Cette fonctionnalité pourrait être considérée comme une amélioration possible, car elle est vraiment pratique d'utilisation mais entraîne un gros travail supplémentaire.\\

Bien évidemment, l'interface graphique proposé peut être largement amélioré : apparence des boutons, modélisation du labyrinthe, mise à jour de la position des joueurs, etc..


\chapter{Gestion de projet}
	% ============= Gestion des documents =============
\section{Gestion des documents}
    Notre équipe de projet était composée de six membres. Nous avons donc dû mettre en place un système de gestion de documents partagés (textes et code source) et de versionnage. Nous avons choisi Git et la plate-forme Github pour cela. \\

    La communication au sein du groupe s'est faite par mail, sms et de vive voix. De plus, l'historique de Git permettait de connaître l'avancement sur le projet de chacun.


% ============= Répartition des rôles =============
\section{Répartition des rôles}
    La spécification a été réalisée à l'aide de l'équipe au complet. Alexandre Brehmer a proposé un début de conception, sur laquelle nous avons tous discuté afin d'apporter des critiques et améliorations. Ces deux phases étant cruciales, toute l'équipe a dû participer. \\

    Une fois la première version de la conception terminée, nous avons pu découper l'implémentation du code en plusieurs tâches. La répartition entre les membres du groupe s'est faite ainsi:

    \paragraph{Alexandre Brehmer} Alexandre a implémenté toute la base de l'architecture RMI\@. Cette base a servi de structure pour la suite de l'implémentation. De plus, pendant toute la phase de programmation, il a fait en sorte de faire fonctionner ensemble les différentes parties de la logique métier.

    \paragraph{Christophe Cluizel} Christophe a implémenté toute la logique métier en lien avec le labyrinthe, à savoir le graphe, le labyrinthe et le générateur (qui a seulement le rôle de charger un labyrinthe à partir d'un fichier texte). De plus, comme Alexandre, il a fait en sorte de faire fonctionner la logique métier dans son ensemble. De façon parallèle au projet, Christophe a implémenté en Scala un outil permettant de générer des labyrinthes automatiquement (sérialisés ensuite en fichier texte). Ce sont ces fichiers textes que nous chargeons au sein de notre application.

    \paragraph{Anthony Courtin} Avec l'aide de Simon, Anthony s'est occupé de l'interface utilisateur. Plus spécifiquement, il s'est chargé de l'interface du menu principal, puis de l'intégration de celui-ci au sein de l'application existante.

    \paragraph{Céline Leduc \& Charlotte Touchard} Céline et Charlotte se sont occupées de l'implémentation de la gestion d'une IA potentiellement créée par l'utilisateur. En effet, une fois le code de l'IA récupérer, il faut pouvoir le compiler et l'exécuter sur chez le client.

    \paragraph{Simon Wallon} Avec Anthony, Simon avait pour tâche de réaliser l'interface utilisateur. En particulier, il a implémenté l'interface ayant un lien avec la gestion du labyrinthe, à savoir l'affichage du labyrinthe et des différents joueurs et les interactions avec l'utilisateur en cours de partie. De plus, il s'est également occupé de l'intégration de l'interface utilisateur avec la logique métier.




\chapter{Conclusion}
Avec ce projet, nous avons réussi à développer un programme mettant en oeuvre de l'informatique répartie. Nous avons pour cela utilisé des notions vues en cours, en utilisant notament le RMI. Nous avons également appris à manipuler d'autres technologies, comme pour l'IHM avec Slick2D.

Le résultat final est un programme fonctionnel en mode fenetré, pouvant donc être utilisé par des utilisateurs non avertis. 

\end{document}
